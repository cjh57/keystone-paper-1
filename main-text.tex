%\documentclass[mathsansserif,xcolor={dvipsnames}]{beamer}
\documentclass[12pt]{article}
\usepackage[hidelinks]{hyperref}
\usepackage[backend=bibtex,hyperref=true,backref=true]{biblatex}
\usepackage{mathtools}
\usepackage[margin=1in]{geometry}
\DeclarePairedDelimiter{\parens}{\lparen}{\rparen}

\usepackage[english]{babel}
\usepackage[dvipsnames]{xcolor}
%\usepackage{hyperref}
%\newcommand{\cjh}[1]{\textcolor{blue}{(cjh: #1)}}
%\newcommand{\stupid}[1]{#1}
%\newcommand{\Ae}{\stupid{\textit{Ae.\ aegypti}}}

\newcommand{\alb}{\textit{Ae.\ albopictus}}
\newcommand{\atl}{\textit{Ae.\ atlanticus}}
\newcommand{\eg}{\textit{e.g.}}
\newcommand{\ie}{\textit{i.e.}}

\newcommand{\cjh}{\textcolor{blue}{cjh}}
\newcommand{\tjh}{\textcolor{red}{tjh}}
\newcommand{\jal}{\textcolor{green}{jal}}
\newcommand{\jgm}{\textcolor{purple}{jgm}}

\newcommand{\msg}[3]{(#1 $\rightarrow$ #2: #3)}

\newcommand{\mcc}[1]{\msg\cjh\cjh{#1}}
\newcommand{\mct}[1]{\msg\cjh\tjh{#1}}
\newcommand{\mtc}[1]{\msg\tjh\cjh{#1}}
\newcommand{\mtt}[1]{\msg\tjh\tjh{#1}}
\newcommand{\mjc}[1]{\msg\jal\cjh{#1}}
\newcommand{\mgc}[1]{\msg\jgm\cjh{#1}}

%MSMR military paper:
%(236+230+1010)/(3205+724+2495) ~= 24% unspecified cause
%2495/(3205+724+2495) to (724+2495)/(3205+724+2495) ~= 39%--50% unhospitalized
%(and probably an underestimate since some of the individuals appear to already
%have been hosp.)

%Principles gives "approximately 20 cases per 100,000 population per year" total
%aseptic mening
%(w/cite), cause cannot be established in "one third or more" of asep men cases
%in "viral isolation centers"

%Khetsuriani gives 36,000 *viral* mening hosp / year average 1988-1999, rate: 14/100,000 * 327.2
%million = 45808 ~= 46,000
%all mening => 27.9/100,000 * 327.2 million = 91288.8 ~= 91,000
%fraction viral = 50.2%
%92% ``unspecified viral meningitis'' (hopefully out of date)
%and 18.2% unspec as to general cause (in all meng)
%3.8% of all mening cases => death; 0.4% of viral mening


%enceph hosp 238,567 (2000--2010)
%enceph death x .056 = 13,359.752
%mening hosp 72000 (in 2006) from holmquist2008meningitis-related
%mening death x .037 = 2664.0


%okay, let's make these a little more comparable:
%enceph hosp: 7.3 / 1e5 population * 327.2e6 = 23885.6 ~= 24,000 per annum
% * .056 = 1337.5936
%mening hosp: 24.1 hosp / 100,000 persons => 78855.2
% * .037 = 2917.6423999999997
%Total hosp: 23885.6 + 78855.2 = 102740.8 "over 100,000"
%1337.5936+2917.6424 = 4255.236 ~= 4300
%no etiology: (7.3*.349+24.1*.172) / (7.3+24.1) = 0.2131496815286624
%"viral" (7.3*.121+24.1*.459)/(7.3+24.1) = 0.38042038216560514

%fraction of viral mening that are only "viral" and nothing more: .459/.546 = 0.8406593406593407

\title%[Keystone Virus in Florida]
{Ecology and Public Health Burden of Keystone Virus in Florida}

\author%[chrishen@umich.edu] % (optional, use only with lots of authors)
{Christopher J. Henry, John A. Lednicky, J. Glenn Morris, Thomas J. Hladish}

\date{\today}

%\bibliographystyle{plain}
\addbibresource{keystone.bib}

\begin{document}
    \maketitle
%    \titlepage

    %This is a placeholder citation, to be removed later.\cite{asdf} %I have placed this citation here to ensure that it is a memorable number 1.
    
%Because the subarachnoid space is continuous around the brain, spinal cord, and
%optic nerves, an infective agent gaining entry to any one part of the space
%allows it to spread rapidly to all of it, even its most remote recesses; in
%other words, meningitis is always cerebrospinal.---from adams & victor's

    %\mtc{intentionally plan in one day off per week}\\
    %\mtc{after intro \& conclusion changes = good time for tom to start editing}\\
    %\mtc{intro not mentioning mosquito spp. is fine}\\
    %\mtc{Phil L. -- useful person to talk to about what is practical & possible in terms of mosquito research in Florida; if i think useful: draft email, send to tom, tom will send with tweaks}
    %\mtc{bias to focusing on problems we have a clear name for}

    \section{Introduction}
        \label{intro}
        %\mtc{Reminder for intro in general: hook, not schematic presentation; tweaking intro emphasis to emphasize high pts of paper one of two current priorities}

        Meningitis, the inflammation of the meninges lining the brain and spinal cord, and encephalitis, the inflammation of the brain itself, are both potentially life-threatening emergencies. In the United States, it is estimated that they collectively account for over 100,000 hospitalizations and 4200 deaths per year\cite{george2014encephalitis,holmquist2008meningitis}, and many survivors experience long-lasting sequelae, which can be severe. The causes of meningitis and encephalitis are diverse, but often poorly determined: 38\% of all cases are diagnosed only as ``viral,'' and an estimated 21\% of cases never receive any diagnosed etiology at all \cite{george2014encephalitis,holmquist2008meningitis}. Keystone virus (KEYV) is a poorly-studied pathogen that may infect many people, and that is related to several other viruses that are known to cause meningitis and/or encephalitis.

        KEYV, genus \emph{Orthobunyavirus}, was named after the location in the Tampa Bay, Florida, area wherein it was discovered. It is a mosquito-borne RNA virus that has a single-stranded tripartite genome, and is presently the only International Committee of the Taxonomy of Viruses-recognized member of the species \emph{Keystone orthobunyavirus} (\ref{asdf}). %\ref{https://link.springer.com/article/10.1007\%2Fs00705-019-04253-6}
        It is a poorly-studied pathogen that infects mammals and that may infect many people in the Eastern USA. KEYV is part of a clade that has historically been known as the ``California encephalitis serogroup'' of genus \emph{Orthobunyavirus}. This clade includes several other North American bunyaviruses that are known to cause meningitis and/or encephalitis, such as California encephalitis, Jamestown Canyon, and La Crosse viruses. Previous studies suggest that approximately 20\% of individuals in the Tampa Bay area had been previously infected with it\cite{parkin1972review}, but the health effects of infection in humans are not well-established. However, other viruses in the same serogroup are known to produce infections that typically are asymptomatic or characterized only by mild, generic symptoms (\eg, fever and rash), but that can also, more rarely result in meningitis and/or encephalitis. For example, La Crosse Virus (LACV), the California encephalitis serogroup virus that is generally considered to be the largest public health threat in the United States, causes encephalitis in only a fraction of infected individuals, with a ratio of encephalitis cases to infections that has been estimated at anywhere from 1 in 3 to 1 in 1500\cite{rust1999topical}. %Some of the works cited may be better sources, and there may be more recent, but I will leave it like this for now

        %This pattern, of a pathogen that causes far more clinically inapparent infections than clinically evident cases, is not uncommon. Outside of the California encephalitis serogroup, it is also true of West Nile Virus, the most commonly diagnosed arboviral cause of encephalitis in the United States. It is estimated that $<1\%$ of human West Nile infections cause encephalitis, approximately 20\% cause nonspecific symptoms such as headache and fever; and the remainder are entirely asymptomatic\cite{cdc2018symptoms}. Likewise, poliovirus, which the global public health community has launched the largest public health campaign in all of human history in order to eradicate, only causes paralytic poliomyelitis in a fraction of infections (among fully susceptible individuals) that varies by strain from 1 in 200 to 1 in 2000. In the vast majority of infections, the only notable symptom is diarrhea.\cite{asdf}

        %\mtc{Add a third, unrelated example, \eg, adenovirus or dengue, just aiming for 1 number, inf vs. severe}
        
        %\mjc{My suggestion: Mention a DNA virus such as JC or BK polyomaviruses. These cause lifelong inapparent infections of humans.}

        Thus, we have numerous cases of meningitis and encephalitis for which the causative species of pathogen is unknown, in Florida as elsewhere in the United States. %\mjc{This statement comes out of nowhere. It looks like a concluding remark, but as presented, is unclear.  `Thus, we have numerous cases....'  Is the intention to point out that many times, an etiologic agent is not diagnosed?}
        We know that other viruses in the California encephalitis serogroup are capable of causing meningitis and encephalitis.  We have evidence of relatively common human infection with KEYV in at least some parts of Florida (including the Tampa Bay area) with KEYV. Is there any evidence of encephalitis cases caused by KEYV itself?

        In fact, there is: following a St.\ Louis Encephalitis epidemic in 1962, serology-based surveillance detected two cases of encephalitis in Florida children that were likely caused by KEYV. As part of that surveillance effort, clinicians at the Encephalitis Research Center of the Florida State Board of Health performed diagnostic tests for hemagluttination-inhibition antibody titers to an antigen that is associated with California encephalitis serogroup viruses. In two cases, they observed a four-fold or greater rise in titers, suggesting infection with a California serogroup virus. In one of the patients, complement-fixation antibodies subsequently developed, providing additional evidence in support of this hypothesis; in the other, they had not been detected by day 94. Both patients presented with encephalitis, one of them severe, although both made apparent full recoveries.\cite{bond1966california}

    These observations were made during the same period in which KEYV was first identified, in Florida mosquitoes.\cite{bond1966california} %While not definitive, this is suggestive \mjc{that . . .}. \mgc{My recollection of the paper is that the child had no travel history to any areas where other California serogroup viruses were present, leading to the assumption that KEYV, which was known to be present locally, was the responsible virus.} %Admittedly, this is not definitive; trivitattus virus, another virus in the California encephalitis virus serogroup, was also detected in Florida during the same period. Based on the viruses detected in mosquitoes in Florida at that time, these cases could have been caused by either Keystone virus or Trivitattus Virus, which is \mcc{roughly equally studied with Keystone; this probably should be either framed differently or made more precise}. \mct{There might also be some plausibility from other evidence sources for LACV? But I don't know that we need to say that, given how cautious we're being here anyway.}
    While this is not definitive for establishing KEYV as the causative agent, it is suggestive.  It is plausible that the infections were caused by trivitattus virus, another virus in the California encephalitis virus serogroup, which was also detected in Florida during the same period. However, there are two lines of evidence that both make KEYV a more likely causative pathogen than trivitattus virus. First, one of the patients exhibited an immune response to infection that resulted in both a faster and a higher rise in antibody titers with respect to California encephalitis virus, \textit{sensu stricto}, (CEV) than with respect to trivitattus virus\cite{quick1965california}). This is consistent with experimental results for KEYV-infected rabbits, but not for trivitattus virus--infected rabbits\cite{jennings1968california}.%
    %suggesting infection with a virus that is more immunologically similar to CEV than it is to trivitattus virus. %is this true of KEYV? probably
    Secondly, KEYV appears to have been more common that trivitattus virus in Florida in that general period\cite{taylor1971california}. %more limited than i'd like but it will do
    %\mcc{Elaborate and fix: One patient exhibited a (probably) far greater increase in antibody titer vs ``CE'' (BFS cross-reactive antigen) than vs trivitattus (<20 -> 80 -> 640 -> 20 vs. <4 -> <4 -> <4 -> 4)}
    
    %Can't find evidence for this: The patients had no travel history to areas where other California serogroup viruses were present.
    
    %When considering the significance of results like this, it is important to bear in mind that both meningitis and encephalitis have the characteristic that a large fraction of hospitalizations (XX\% for meningitis, 47\% for encephalitis\cite{george2014encephalitis}) either result in no identified etiology, or an etiology that is identified only as ``probably viral.'' There is further evidence that many of these wholly or partially unidentified cases are likely due to arboviral (insect- or tick-transmitted virus) infections, based on seasonality. Combined with evidence of a relatively high (circa 20\%) human lifetime incidence of human KEYV infection in Florida, this raises the possibility that (undiagnosed) Keystone virus may be a substantial contributor to meningitis and encephalitis cases in Florida.

        Assessing the public health impact of KEYV in Florida, and what can be done to combat it, requires answers to questions about basic science, health effects, costs, and prognoses. As many of these questions are currently unanswered, and some of them even unasked, the first step in answering them is to enumerate what they are. Broadly speaking, they can be grouped into four categories:
        \begin{enumerate}
            \item ``What are the basic transmission dynamics of KEYV, and how, if at all, are they currently changing?''
            \item ``How many people are infected with KEYV each year in Florida?''
            \item ``What is the probability of an infection resulting in each acute disease state (\ie, none (fully asymptomatic), generic symptoms (\eg, fever and rash), meningitis without encephalitis, and encephalitis or meningoencephalitis)?''
            \item ``What is the expected cost---in dollars, in disability, and in pain and suffering---of each of those acute disease states, when caused by KEYV?''
        \end{enumerate}

        In the following sections, we review the current state of scientific knowledge with regard to each of these questions, and what important gaps remain in our collective understanding.

    \section[Transmission dynamics]{Question 1: What are the transmission dynamics, insofar as they are relevant to disease burden and/or control?}
    \label{transmission-dynamics}
        The potentially relevant sub-questions here can be broadly split into two categories. First, there are questions of basic science, such as ``Which mosquito species are competent vectors for KEYV,'' ``How large of a role do mosquito-mammal-mosquito transmission cycles play in sustaining enzootic Keystone infection among mosquitoes,'' and ``Can enzootic Keystone infection among mammals be sustained during periods of the year in which mosquitoes vectors are absent?'' These are important questions because they lay essential groundwork for the second category of sub-questions. Second, there are questions that relate to the potential for, and cost-effectiveness of, intervention at this time. These include questions such as ``Who is at risk, and to what extent,'' ``What, if any, interventions are possible,'' and ``How much are each of those interventions likely to cost?''

        Regarding the basic science, let us start with what we already know:

        \subsection{California serogroup viruses}
            \label{california-serogroup}
            KEYV belongs to the California serogroup, a clade of viruses in the genus \textit{Orthobunyavirus}. This serogroup derives its name from its earliest-discovered member, California encephalitis virus (CEV). All of the Viruses in this serogroup have been observed in mosquitoes. Those for which transmission has been observed or inferred all exhibit a two-host cycle, involving mosquitoes and mammals. (In most cases, the mammals involved are small rodents or lagomorphs, although deer are the primary mammalian reservoir Jamestown Canyon virus.) Thus, they fall into the broader category of viruses transmitted by arthropod vectors, known as \textit{arboviruses} (an abreviation of ``arthropod-borne viruses'').
        
        \subsection{Keystone virus and its hosts}
            \label{california-keystone}
            \subsubsection{Mosquito terminology}
                \label{mosquito-terminology}
                In 2000, the subgenus \textit{Ochlerotatus} of the genus \textit{Aedes} was elevated to the genus level. As a result of this split, the mosquito species historically known as \atl\ now belongs to the genus \textit{Ochlerotatus}\cite{reinert2000new}. Some sources identify the mosquito species historically known as \alb\ as a member of this genus, others as a member of the subgenus \textit{Stegomyia} within \textit{Aedes}, and others still regard \textit{Stegomyia} as a full genus in its own right.
                
                Because of historical usage, it is still very common to see these species, and other species that were formerly classified as \textit{Aedes}---or whose classification in \textit{Aedes} is currently a matter of debate---referred to as \textit{Aedes}, even by those who accept the reclassifications. For this reason, in this paper, we will refer to these mosquitoes as \textit{Aedes} species.

                The mosquito species \textit{Ae.~atlanticus} and \textit{Ae.~tormentor} are extremely difficult to distinguish as adults, are closely related and biologically similar, and have similar ranges\cite{burkett2013mosquitoes}. They are therefore frequently lumped together in mosquito surveys, sometimes described as \textit{Aedes atlanticus tormentor}\cite{bond1966california} or \textit{atlanticus-tormentor}, or, sometimes, simply as \atl\, with a note indicating that ``\atl'' pools may include both species. Other research may distinguish between the two, or may be conducted in areas where only one is present. For the purpose of this review, we will not distinguish between these cases; for the sake of brevity, we shall allow the term ``\atl'' to encompass both species.

                The mosquito species \textit{Ae.~infirmatus} is less similar to \textit{Ae.~atlanticus} and \textit{Ae.~tormentor} than they are to each other, but is still similar enough to occasionally be lumped in with them as \textit{Ae.~atlanticus-tormentor-infirmatus} when fine distinctions between the three are not important. In the case of KEYV, however, the distinction appears to be important, with \textit{Ae.~infirmatus} showing a much lower prevalence of infection than \textit{Ae.~atlanticus-tormentor}.\cite{taylor1971california} Therefore, we maintain this distinction.

            \subsubsection{Keystone virus}
                KEYV was first isolated from mosquitos found in 1962 in the Tampa Bay area during a St.\ Louis Encephalitis epidemic\cite{chamberlain1969arbovirus, taylor1971california}, and was first identified as a distinct strain of California encephalitis virus (as it was then classified) in 1964\cite{bond1966california}. It is primarily transmitted by the floodwater mosquito \atl. Another floodwater mosquito, \textit{Ae.~infirmatus}, appears to play a lesser, but perhaps still meaningful, role in transmission\cite{taylor1971california}. The prevalence of KEYV in female \atl\ has consistently been found to be approximately 0.003 across several surveys, conducted in different years and different locations\cite{watts1988maintenance,taylor1971california,leduc1975ecology,chamberlain1969arbovirus}. The greatest amount of research on both vertebrate hosts and transmission dynamics of KEYV was done in the Pokomoke Cypress Swamp in Maryland, in the 1970s and 1980\cite{watts1988maintenance}. In that region, the primary vertebrate hosts of KEYV are rabbits and grey squirrels\cite{watts1988maintenance}. A smaller amount of research on the same subject, conducted in Florida, supports a major role for rabbits and grey squirrels as well.\cite{jennings1970tamiami,jennings1968california} In addition, there is some evidence that cotton rats may also be important hosts in Florida\cite{jennings1970tamiami,taylor1971california}; cotton rats are not found as far North as the Pokomoke Swamp\cite{watts1982serologic}. In all of these primary hosts, KEYV appears to produce a 1--5 day viremia\cite{watts1988maintenance,jennings1968california}, with generally mild symptoms\cite{asdf}. %florida rabbit and squirrel cites are kinda weak; watts1982serologic cite is imperfect

                A number of other hosts, including whitetail deer, horses, cattle, dogs show a high seroprevalence of neutralizing antibodies to KEYV when they are raised in enzootic areas, but do not appear to develop detectable viremia, and are therefore highly unlikely to play a significant role in transmission dynamics.\cite{parkin1973occurrence,watts1982serologic,watts1979experimental} While this may reduce their interest from a purely transmission modeling standpoint, it raises the potential, particularly for domestic animals, of their use as sentinels for surveillance.\cite{parkin1973occurrence}

            \subsubsection{Pokomoke Cypress Swamp results}
                \label{pokomoke-results}
                \mct{This sub-sub-section is still quite rough}

                Research on the ecology and transmission dynamics of KEYV conducted both in the Pokomoke Cypress Swamp and in laboratories on mosquitoes collected there generated more results than it is practical to summarize here. But particularly notable for our purposes, apart from the basic results about species involved, are the following: Viremia and seroconversion in squirrels occurred only when \atl\ were present.\cite{watts1988maintenance} Individual squirrels were tracked, and some squirrels were observed to serorevert, \ie, to become seronegative after being seropositive. Of these, all who were recaptured again late in the \atl\ feeding season were seropositive again, suggesting reexposure and perhaps reinfection; however, circumstantial evidence suggested that this reinfection, if any, was not accompanied by a second viremia\cite{watts1988maintenance}, consistent with previous results\cite{watts1979experimental}.
                
                Orally-infected \atl\ were minimally effective at transmitting Keystone to suckling mice they fed upon: Out of 82 bites, only 1 produced a lethal infection (the primary endpoint considered; Keystone infection has a high lethality in suckling mice); 50 of the surviving mice were tested for antibodies, and only 7 were positive.\cite{watts1988maintenance} \mjc{Be careful. Suckling mice, which do not have a fully formed immune system, may not respond well.   So be sure to add more info to explain this experiment...or people will challenge the validity of your statement.} Evidence from previous dose-response experiments suggests that these 7 seropositive mice likely received very low doses of KEYV, and would have been minimally viremic at most. In contrast, 33 out of 46 mosquitoes \mgc{In a modern context, this type of work is likely to be considered questionable or wrong: infected by intrathoracic innoculation} produced lethal infections in mice they were fed on\cite{watts1988maintenance}, with 6 of the remaining 13 producing seroconversion in the fed-upon mice. Given this discrepancy, and evidence of rapid infection of sentinel rabbits exposed during seasons of peak \atl\ activity\cite{jennings1968california,leduc1978natural}, this raises the possibility that transovarially-infected (vertically-infected) \atl\ may be substantially more efficient transmitters than horizontally-infected \atl. \mjc{This explanation may not be viewed favorably in a modern context. What matters is whether the virus is in the salivary glands, and also, the amount of virus therein.} \mgc{I agree – be careful with this statement – concepts are somewhat out of date.}
            
            %\mcc{Possibly stick a mention of the TOT results in watts1988maintenance here}

                However, the Pokomoke Cypress Swamp is hardly a perfect model for transmission dynamics in Florida, for a number of reasons. As already noted, there are no cotton rats in the Pokomoke Swamp, meaning that any contribution these make to Keystone transmission dynamics in Florida will not be observable there. It is near the northern range limit of \atl, and the season of \atl\ activity is accordingly shorter.

        \subsection{Routes of transmission of KEYV}
            \label{transmission-routes}
            Since the initial discovery of KEYV, it has been believed to be transmitted horizontally, in a two-host cycle, mosquito-to-vertebrate and vertebrate-to-mosquito. This is indeed supported by experimental evidence. What was not realized until substantially later is that KEYV (along with various other California serogroup viruses) was also transmitted vertically, from an infected female mosquito to a fraction of her eggs. There is evidence, as regards other California serogroup viruses, that vertically infected mosquitoes are themselves more capable of vertical transmission than horizontally infected mosquitoes, a phenomenon known as \textit{stabilized transmission}. Somewhat more surprisingly, there is also evidence that vertically infected mosquitoes may be more efficient horizontal transmitters as well. Indeed, horizontally-infected \atl\ appear to transmit quite poorly, with one experiment (as noted above) finding only 1 definite transmission out of 82 bites.\cite{watts1988maintenance}

            There is also some circumstantial evidence supporting the possibility of within-species horizontal transmission, both among mosquitoes (during copulation)\cite{asdf} and among cotton rats (through urine and/or bites)\cite{taylor1971california}. Neither of these, however, is confirmed. A weakness of all previous studies, however, is that the mosquitoes used in experiments may have been aberrant strains.  For example, wild-type \emph{Ae.\ aegypti} from Orlando, Florida, differ in their ability to transmit dengue virus compared to \emph{Ae.\ aegypti} laboratory strains. \mtc{cite}

        \subsection{The uncertain role of \textit{Aedes albopictus}}
            \label{albopictus}
            \alb, the Asian tiger mosquito, is an invasive species that has been spreading in the United States since its introduction into Houston in 1985, most likely in a shipment of used tires. It is more closely associated with humans than \atl, has a larger range, and is present for a greater portion of the year. Most of the work on prevalence of KEYV infection in different mosquito species was done before the arrival of \alb, either in the region where the work was being done, or in the US as a whole. Multiple studies have since made it clear that \alb\ is at least occasionally infected with KEYV. This raises the question of whether it could play a role in increasing the enzootic range of KEYV and/or the frequency with which humans in that range are infected.

            However, another study, by Grimstad \textit{et al.} \cite{grimstad1989recently} has been cited, directly or indirectly, by a variety of review papers\cite{asdf,asdf} as providing evidence that, while \alb\ is readily capable of being infected with KEYV, it is at most very poorly capable of transmitting KEYV to vertebrates in turn. \mtc{ check wording of following} However, in similar experiments, \atl\ has produced similar results, despite being a competent vector.  These results taken together suggest that this experimental protocol may be inadequate for assessing vector competence.  See SI for further details about these experiments.
            
        \mtc{Move the following to an SI}

        Grimstad \textit{et al.} fed 37 uninfected, mature female \alb\ on a mixture of KEYV and rabbit blood, allowed 14+ days for an incubation period, re-fed each on a suckling mouse, and ground up their midguts and heads (separately) and tested for the presence of KEYV. Transmission to mice was assessed by observing them for signs of disease, euthanizing any who showed such signs, and then testing their brains for signs of KEYV. The researchers found that 31 of mosquitoes developed a disseminated infection, and another 3 were infected at the midgut level, but that none of them transmitted virus to the mice they fed on. Thus, the result of this study was that none of the 34 horizontally infected \alb\ produced an infection in suckling mice that was lethal or symptomatic enough to be observed by the researchers.

        As noted in Section \ref{pokomoke-results}, Watts \textit{et al.} horizontally infected 82 \atl, re-fed each on a suckling mouse, and observed only one lethal infection. When they tested 50 of the 81 remaining mice for antibodies to KEYV, 7 were positive, suggesting a possible mild or asymptomatic infection; if those 50 mice were a representative sample of the 81, we can estimate that approximately 11 of the 81 mice that did not experience a lethal infection may have experienced a non-lethal infection. We cannot know what fraction, if any, of these mice displayed sufficient signs of morbidity that they would have been euthanized had the protocol of Grimstad \textit{et al.} been used instead. If there would have been some, we also cannot know what fraction of those, if any, would have have detectable virus in their brains, given that there is evidence that not all antibody-producing infections necessarily result even in detectable virus in the blood\cite{watts1988maintenance}. So this result may reasonably be considered equivalent to a result of somewhere from 1 to 12 detected transmissions out of 82 opportunities using the methods of Grimstad \textit{et al.}, without certainty as to where in that range it lies. Comparing this range of possible results to Grimstad \textit{et al.}'s result of 0 detected transmissions out of 34 opportunities using a Fisher's exact test, we find that the p-value ranges anywhere from 1 (i.e., no evidence at all that the probability of transmission is different) at an assumed 1 detected transmission for \atl, down to 0.0174 (statistically significant, but not highly significant) for an assumed 12 detected transmissions; the conventional cutoff of $p < 0.05$ is achieved for an assumed 10 detected transmissions (i.e., 9 out of a probable 11 non-lethal infections resulting in both sufficient morbidity to be observed by the researchers, and sufficient virus in the brain to be detected), but not at an assumed 9. In any event, this is a p-value for whether the transmission rates differ \textit{at all}, not whether \textit{Ae.~albopictus}'s transmission rate is so much lower as to make it effectively irrelevant as a vector.

        In any event, as noted in Section \ref{pokomoke-results}, the transmission rate from horizontally-infected \atl\ is likely much lower than the transmission rate from vertically-infected \atl. Thus, the really relevant questions for whether \alb\ is capable of being an important vector for KEYV are likely to be how well \textit{vertically}-infected \alb\ transmit KEYV, both vertically and to vertebrate hosts. This question does not appear to have been studied.
        %these studies have exclusively \mcc{?} looked at horizontally-infected \alb, and have (implicitly) compared the poor transmission observed to either ecological transmission by all \atl, regardless of how they were infected, or ``transmission'' to suckling mice by direct cerebral injection of pureed \atl. Given evidence of poor transmission by horizontally-infected \atl, this is clearly a poor comparison. Testing whether \alb\ can be vertically infected with KEYV and, if so, how readily vertically-infected \alb\ transmit would be an obvious next step in assessing the potential for \alb\ to play a non-trivial role in KEYV transmission.

    \section[Incidence of infection]{Question 2: What is the incidence of (relevant) infection?}
        \label{incidence}
        %\mtc{Wouldn't frame as prac + phil}
        %\mtc{strike ``relevant'', (``& we define it as'' in my notes -- note
        %sure what that meant)
        %\mtc{Context specific econ. vs. trans. analysis vs. clinical}

        There are a number of sub-questions to address, regarding the incidence of infection. The first is what exactly is meant by a ``relevant infection.'' Partly, this is a philosophical question, and partly a practical one. In a philosophical sense, there is the question of what constitutes ``an infection'' in the first place. In a practical sense, there are both the question of what (1) what definition of a ``relevant'' infection will allow us to (comparatively) easily characterize the probability that a given individual will experience one, and (2) what definition will result in all relevant infections having a (at least, conditional on patient characteristics not related to a history of KEYV exposure) a similar risk profile. Ideally, we want a definition that will encompass all or almost all exposures to KEYV that have a non-negligible expected cost, that will have well-defined risks associated with it, and that will be feasible to measure.

        For some pathogens, there would also be a question of whether we want to measure incident or prevalent infections. In practice, this distinction is likely to be of little importance for Keystone, for reasons that also imply that direct measurement of infections is likely to be realistically impossible at the present time. In vertebrates believed to be reservoir hosts, viremia lasts for less than 1 week, and likely occurs only once in a lifetime\cite{watts1988maintenance,watts1979experimental}. Keystone appears to rarely present with specific symptoms in humans, and so it would be challenging to narrow down the population further, attempting to obtain an reliable direct estimate of the number of incident or prevalent infections would be cost-prohibitive.
        
        %If we make a liberal assumption that the measured (approximately) 20\% seroprevalence represents the fraction of the population infected by age 20, this would still imply that the probability of being currently infected, for individuals under age 20, is on the order of 1 in 5000.

        However, there is an alternative approach that we can use, looking at \textit{seroprevalence}, \ie, the prevalence in the human population of neutralizing antibodies to KEYV at titers that are likely to indicate a past infection by the virus. To assess this, multiple \textit{serosurveys} have been performed in the Tampa Bay area, and have consistently found a seroprevalence of approximately 20\%\cite{parkin1972review}.% \mcc{elaborate}
        %parkin1972review actually just says ``Florida'' and cites 150 and 151
        %also cites a total fail in KEY trans test in atl to 150
        %and infirmatus infected to 118 (penny1969penetration)
        %rabbit 40\% to 150, rat 17.x\%
        %horse 25-48.6\% and cow 0-36.x\% 150,116
        %dog 12.3\% 116
        %innoculation: rabbit viremia 80 (jennings1968california)
        %c.rat vir. 150
        %innoculation: dog and horse Abx 117
        %150. Wellings, F. M., 1968. Epidemiological-virological study of the California encephalitis virus group in Tampa Bay area of Florida 1963--1968. D.Sc. (Hyg.) Thesis, University of Pittsburgh.
        %151 Wellings personal communication
        %at least those two do imply Tampa Bay (for 150 at least)
        %117 Parkin California Encephalitis Virus in domestic mammals II
        %Inoculation with Keystone and Trivitattus virus. In preparation.


        While using seroprevalence data to make inferences about past incidence allows us to get around the expense of attempting to measure incidence directly, it introduces several problems of its own. First, it is not clear that all infections that carry a meaningful risk of severe negative health outcomes will necessarily result in seroconversion if the infected individual survives. Indeed, one would imagine that immunosuppression (due to HIV, drugs, genetic defects, etc.) would both increase the risk of severe outcomes and decrease the likelihood of generating neutralizing antibodies (if one survives). Anyone who dies as a result of infection will not be measured by a later serosurvey either, although the expected bias introduced by this may be mitigated by the expected low mortality rate. There is also evidence from studies of transmission to grey squirrels in the Pokomoke swamp\cite{watts1988maintenance} that initially seropositive squirrels can revert to seronegativity; if this is applicable to humans as well, then seroprevalence will underrepresent the cumulative incidence of having had at least one infection resulting in seroconversion. The same is true if (re-)infection is possible despite seropositivity. However, given evidence that reinfection appears to (at least in animal models) produce minimal or no viremia\cite{watts1988maintenance,watts1979experimental}, it may not be relevant from a transmission dynamics perspective; if rapid containment by the immune system renders it sufficiently negligible in its risks, it may also be irrelevant from a public health standpoint. 
        %\mcc{Tweak and add on re-finding source: On the other hand, some unexplained deaths of research animals might suggest the possibility of an uncommon but dangerous hypersensitivity reaction\cite{asdf}.}

        On the other hand, cross-reactivity with antibodies induced by other California serogroup viruses could in principle lead to seroprevalence \textit{over}-estimating the cumulative incidence of Keystone infection.

        In addition to addressing the above concerns, in order to accurately estimate the incidence of relevant KEYV infections in the state of Florida using serological data, it would also be necessary to perform a serosurvey for the state as a whole mcc{incorporate more smoothly: which would in turn require determining how to treat regular visitors and part-year residents}), or some reasonably representative subset thereof, rather than solely for the Tampa Bay area. In doing so, it would be highly desirable to record the seroprevalence not only overall, but also by age. This would offer additional data to be included in a model of patterns of seroconversion (taken as a proxy for infection, with the appropriate caveats) and, if applicable seroreversion. However, care must be used in the interpretation of such data if infection rates are changing over time, \eg, because of changing patterns of land use, or increased transmission due to the spread of \alb\ or other invasive mosquito species.

        %\mcc{``no tests for keystone'' in my hardcopy notes}

        \subsection{Recent case}
            \label{recent-case}
            While the serosurvey results are certainly suggestive, until recently, there had been no confirmed human cases of current KEYV infection. In 2016, however, a 16-year-old male presented at a Northern Florida urgent care clinic with symptoms of low fever, mild fatigue, and a mild rash. These are mild, common, and non-specific symptoms in children and adolescents, which often receive no diagnosis beyond ``it's a virus.'' In this case, however, an unusually extensive workup was performed due to concerns about possible Zika virus infection, as there was an outbreak of Zika virus disease in the Americas and Caribbean at that time. As a result, KEYV was detected in the patient's urine.\cite{lednicky2018keystone}
        

        \subsection{Is Keystone virus common in Florida?}
            \label{raises-the-question}
            Combined with serosurvey data, and what we know about other California serogroup viruses, this raises the question of whether KEYV is in fact a relatively common cause of childhood infections in Florida, that is severely underdiagnosed for multiple reasons. First of all, infections may generally be asymptomatic. Moreover, evidence from both presumptive reservoir hosts and the one directly observed human infection to date all point to infections that, when they do present with symptoms, present most frequently with mild and non-specific symptoms that are unlikely to provoke anywhere near the level of effort that was involved in obtaining a specific etiology for the 2016 case. Presumptively viral fever and rash are often not being worked up beyond exclusion of major pathogens of concern.
        
            Furthermore, it may be difficult to detect infection even when looking for it: In the 2016 case, after screening of both saliva and urine for a variety of viral RNAs using RT-PCR came up negative, a more intensive, generic screening procedure yielded a positive result for KEYV.  In the second test, the patient's saliva and urine were screened by amplifing any viral DNA or RNA present, using PCR and RT-PCR, respectively, followed by TA-cloning into plasmids, and Sanger sequencing\cite{lednicky2018keystone}. This process is sufficiently difficult and labor-intensive that, unlike more straightforward screening procedures, it is not practical for routine diagnostic use.\cite{lednicky2019personal}

            Of 20 plasmids containing inserts amplified from urine and 20 containing inserts amplified from saliva, only 1 (from urine) contained Keystone cDNA. Simply by chance, it could have happened that no plasmids contained Keystone cDNA, and detection would therefore not have occurred. Note that blood was not tested, and it is not known what biological sample or timing of collection would be best for detecting the virus.
            
            Moreover, because this approach depends on finding a match for the sequencing results in the GenBank sequence database, it could not have identified Keystone infection, even in principle, prior to the matching sequences being added to GenBank.  The earliest (U12801.1) was added in 1995\cite{genbankU12801.1}, with the two others matches (KT630293.1 and KT630290.1) added only in 2015\cite{genbankKT630293.1,genbankKT630290.1}.

            If KEYV infection is in fact a reasonably common childhood infection in Florida, unrecognized as such until now, this raises the question of whether its public health significance has been likewise underestimated.


    \section[Probabilities of disease, given infection]{Question 3: What are the probabilities of various acute disease states, conditional on infection?}
        \label{probabilities}
        As discussed in the previous section, it is likely that the most common acute disease state caused by KEYV in humans is fever, or fever and rash, without any more severe symptoms. This is still not necessarily innocuous, even assuming that the fever is not high enough to be dangerous. It is still a potential source of numerous costs, financial or otherwise, as discussed in the next section. Nevertheless, its expected harms are much less than those of some other plausible outcomes.

        In particular, several American species within the California serogroup are known to cause meningitis and/or encephalitis in humans, including CEV, Jamestown Canyon Virus (JACV), and, La Crosse Virus (LACV); the latter is believed to be the second most common arboviral causes of encephalitis in the United States today, after West Nile Virus (WNV).

        %\mcc{Move expected cost bits of the following to the next section}
        \subsection{Meningitis and encephalitis}
            \label{m-and-e}
            Meningitis and encephalitis are a pair of inflammatory conditions of the central nervous system of variable etiology and prognosis.
            
            \subsubsection{Meningitis}
                \label{meningitis}
            Meningitis is an acute inflammation of the \textit{meninges}, the 3 membranes that surround the brain and spinal cord. The classic symptoms of meningitis are headache, fever, stiff neck, and photophobia; however, all of them vary substantially in frequency depending on the underlying etiology. The prognosis also varies greatly by etiology, including as regards the probability of hospitalization and expected duration of and cost of hospital stay, the probability of complications, the probability of mortality, and the risk of long-term sequelae.

        \subsubsection{Viral meningitis}
            \label{viral-meningitis}
            Over half of all meningitis cases in the modern US are viral in nature. These are generally less severe than bacterial or fungal meningitis, to the point that some sources refer to viral meningitis with descriptions such as ``generally a benign and self-limited condition''\cite{khetsuriani2003viral} or even simply ``a self-limiting condition''\cite{balada2019cost}. Because of this comparative mildness, it is believed to be heavily underdiagnosed. Numerous different genera of viruses can cause meningitis, but the most common in the modern US are various non-polio enteroviruses. For meningitis caused by some of these pathogens (herpes simplex, varicella zoster, and cytomegalovirus, all of which are herpesviruses), there are specific treatments. But for most, only supportive care is possible. This fact, combined with a generally mild course of disease that often does not require supportive care, results in many viral meningitis cases that are diagnosed being diagnosed in an outpatient setting, without subsequent hospitalization, or attempts to identify the specific agent beyond determining that it is not a herpesvirus. As a result, an estimated $84\%$ of diagnosed viral meningitis cases do not receive a more specific diagnosis\cite{holmquist2008meningitis}. Naturally, such cases will not be included in the statistics for the actual causative pathogen, and will therefore contribute to underdetection.

        \subsubsection{Encephalitis}
            \label{encephalitis}

            Encephalitis is an inflammation of the brain itself, generally associated with altered mental status and other neurological symptoms. In-hospital mortality averages 5.6\% in the U.S., but varies immensely by cause. For survivors, long-term sequelae, sometimes severe, are more common than with meningitis. Approximately 35\% of all cases of encephalitis in the US have no identified etiology at all\cite{george2014encephalitis}.
            %Over half all of cases of encephalitis in the US have no identified etiology. %\mcc{``Over half'' contradicts \cite{george2014encephalitis}: $83,212/238,567=0.34879928908859986< 0.5$, though $> 1/3$. Where did I get ``over half'' from?}
            %Never mind, they're defining unidentified to include viral NOS;
            %i'll recalulate.
            
        \subsubsection{Viral encephalitis}
            \label{viral-encephalitis}
            %fraction of all cases diagnosed as viral: (32918+13698+950+1313+1019+3513+7612+28788) / 238567 = 0.3764602815980416
            %fraction of etiological cases that are viral: (32918+13698+950+1313+1019+3513+7612+28788)/(238567-83212) = 0.5781017669209231
            %fraction of viral cases that have no further diagnosis: 28788/(32918+13698+950+1313+1019+3513+7612+28788) = 0.32053980024718576
            %fraction of all cases that are "viral", nothing further: 12.1%
            Approximately 38\% of all encephalitis cases are diagnosed as viral encephalitis, amounting to 58\% of the cases that receive any etiological diagnosis at all. Approximately 32\% of the cases that are diagnosed as viral encephalitis (amounting to approximately 12\% of all encephalitis cases) receive no more detailed diagnosis. As with viral meningitis, this contributes to underdetection. Overall mortality for viral encephalitis is 8.2\%, but ranges from $<1\%$ for some pathogens---including La Crosse Virus, a close relative of KEYV---up to effectively 100\% for some rare causes, such as rabies\cite{george2014encephalitis}.

            Historically, herpes simplex encephalitis has been near the top end of that spectrum. It is the most common cause of viral encephalitis in the United States, and, if untreated, results in approximately 70\% mortality, with approximately 97\% of survivors experiencing severe, long-term sequelae. Current treatment methods have reduced these numbers, in hospitalized patients, to somewhere in the neighborhood of 10\% mortality, and 75\% of survivors suffering from severe, long-term sequelae\cite{bradshaw2016herpes}. While a vast improvement, this is still a horrifying prognosis.
            %Exact statement in source is 5-15\% and 69-89\% I think there may
            %have been another source I originally examined as well, but this
            %will do for the moment
            
            Because of this wide variability, and the substantial overall mortality, ``viral encephalitis'' is not as salient of a conceptual category as ``viral meningitis''.

        \subsubsection{Meningitis/Encephalitis spectrum}
            \label{spectrum}
            The line between meningitis and encephalitis can be somewhat blurry. Many of the same pathogens cause each; this is particularly true for viruses. Sometimes, both conditions can be present simultaneously, a condition known as \textit{meningoencephalitis}. Even when this is not the case, symptomatic diagnosis can be tricky. Meningitis can present with minimal or no neck stiffness, and with alterations in mental status. Likewise, encephalitis can present with symptoms that include the full classic meningitis triad, and with minimal or no detectable alterations in mental status. Additionally, a viral infection of the meninges can in some cases produce inflammation of adjacent portions of the brain itself---which is, by a literal definition, encephalitis---without the brain itself being infected, which is what ``viral encephalitis'' is normally understood to mean\cite{ropper2014adams}.
        
        \subsubsection{Reporting}
            \label{reporting}
            Neither meningitis nor encephalitis as such is a reportable disease in Florida, but bacterial and fungal meningitis both are, as are amoebic encephalitis and HSV encephalitis. Additionally, all ``arboviral disease'' is reportable, whether under more specific labels (including St. Louis encephalitis, Venezuelan equine encephalitis, Eastern equine encephalitis, and ``California serogroup virus disease'') or under the catch-all category ``Arboviral diseases not otherwise listed''\cite{florida2016reportable}. Of course, such reporting cannot occur without specific diagnosis.

        %\subsection{Others?}
        %    \mct{Is this a section worth including. \eg, ``unknown unknowns'' and/or reference what is known about other serogroup viruses?}

    \section[Expected costs, given diseases]{Question 4: What is the expected cost of each acute disease state?}
        \label{costs}
        Having considered questions about the various acute disease states that may (potentially) follow KEYV, and how likely experiencing each one is, it remains to consider the cost of each. It is important to recognize that the costs of disease are not exclusively financial; pain and suffering, both acute and chronic, may result as well, and should certainly be considered a ``cost.'' The same is true of disability---again, both as experienced on a short-term basis during the disease state itself, and with regards to impairments resulting from long-term sequelae.

        %Nevertheless, it is conventional to express the sum of all these different types of costs in dollars. There is no reason that one must do so. But when resources are limited---which is to say, in practice, always---decisions about how best to allocate those limited resources will be most effective if all costs and benefits are expressed can be expressed in the same units. These units do not have to be dollars. But most costs of interventions are expressed in dollars to begin with, as are many of the costs of disease (and hence, the benefits of interventions that avert disease). Moreover, these costs all already expressed in readily-understood units (dollars, or other local currency, when considering interventions in other parts of the world), which is not true of pain or disability.
        
        %Therefore, by converting these other costs into dollar amounts, we can readily account for them in a cost-benefit analysis in way in which we could not otherwise. The exact dollar values that should be placed on them are apt to debatable at best. Indeed, there is a great deal of variation in the values assigned to a given disability not only by different researchers and organization, but by the same organization at different times, to the extent that the correlation (on a logit scale) between the disability weights (see next section) used by the WHO in official reports in 2004 and 2010 is only $r = 0.61$!\cite{GlobalDALYmethods_2000_2011}

        %Looked at one way, this is frustrating, and somewhat disturbing. Looked at another, it amounts to disagreements about the relative undesirability of different negative outcomes being pushed into the open in a way that they likely would not be if a less quantitative method of attempting to account for all of those outcomes in cost-benefit analyses were used---if such an attempt were made at all. In practice, the most common consequence of not attempting to treat all negative outcomes as commensurable is simply that some negative outcomes do not get weighed into decision-making at all.

        %\subsection{Overview of Quality-Adjusted Life Years (QALYs)}
        %\mtc{I don't necessarily think you need to explain QALYs vs DALYs.  How to do health economics is not within the scope of a paper about KEYV.  I know you're putting this in because of something I said previously, but I think the context has changed--when you're giving a talk, and you know a good chunk of your audience doesn't know a specific methodology, it's worth going on a tangent.  In a paper, you just cite an appropriate resource.  Let's discuss.}
        %    \mct{Shifting terminology slightly based on history of the terms; ``QALYs'' is older and seems at least arguably more general (and to the extent that it's not more general, but strictly different, aligns more with my philosophical views anyway).}
        %    Although this paper does not focus on the methods used to perform this conversion, it is perhaps useful to briefly summarize them, as they are commonly used today. At heart, it is a two-step process: Convert all non-economic costs of negative health outcomes (essentially, deaths or reductions in individuals' quality of life) into a common unit, and then equate that unit to a number of dollars in order to render it commensurable with economic costs. One such common unit, or family of common units, is Quality-Adjusted Life Years (QALYs) lost.
            
        %    \mct{Not sure if I need some statement like this: ``Depending on exact definition, Disability-Adjusted Life Years (DALYs) may be a form of QALYs, or a related-but-distinct unit. But this summary will not get into such details.''}

        %    Broadly speaking, the number of QALYs lost due to a single death is equal to the dead person's life expectancy in years if that death had not occured, multiplied by a factor representing their expected quality of life under that same counterfactual. Similarly, the number of QALYs lost due to living with pain, disability, or some other quality of life--reducing condition is equal to the number of years living with that condition, multiplied by a factor representing the expected reduction in quality of life due to living with that condition, relative to living without it. How these factors should be calculated, and whether additional factors should be incorporated, such as time discounting for events that are expected to occur years into the future, are the subject of much debate.

        %    In any event, once all non-economic loses are expressed in QALYs, expected values can be taken in order to determine the expected loss of QALYs due to each possible outcome. These, in turn, can be summed up.

        %    Finally, there is the matter of equating a lost QALY to some number of dollars. In practice, this can be done by examining how much more a healthy individual has to be paid in order to take a job with an increased risk of death. \mcc{Give some sample values.}

        \subsection{Contributing factors to the expected cost of each acute disease state}
            \label{cost-factors}
            There are a number of both economic and non-economic costs to consider for each acute disease state. On the economic side, in the short term, there are hospitalization costs, the cost of missed work or school (for the patient and potentially also for parents or other relatives). In the longer term, there may be costs associated with various sequelae. On the non-economic side, there is the risk of death, impairment during the acute condition, pain suffered during the acute condition, the risk of impairment and/or pain due to long term sequelae, and heightened long-term risk of death (if any).

        \subsection{Fever and rash}
            \label{cost-fever}
            Fever and rash, without further complications, is the likely most common consequence of (symptomatic) Keystone infection, and is also the least costly. But it is not altogether without cost. There is the cost of suffering, both by the infected individual and, in the pediatric case, by the infected child's parents. While this will in general be far less severe than the cost in suffering inflicted by a case of meningitis or encephalitis, it is hardly negligible---as can be attested by any parent who has had to deal with a feverish child. There is also, in many cases, a cost in missed work for an infected adult, or for the parent of an infected child; likewise, the infected child (or adolescent, or young adult) themselves often incurs a cost in missed education.

            Moreover, there is the cost of any medical services that may be required to ascertain that the cause of the fever and rash is not an emergent or readily communicable one. If antibiotics are prescribed, on a ``precautionary principle'' or to placate the patient or parent, there is not only the cost of the antibiotics themselves, to the patient and/or their insurance provider; there is also a health cost to the patient in the disruption of intestinal microflora, and a whole slew of costs to society as a whole due to (aggregated across numerous patients) an increased risk of the development and proliferation of antibiotic-resistant bacteria.

        \subsection{Meningitis}
            \label{cost-meningitis}
            As noted in Section \ref{viral-meningitis}, viral meningitis is generally regarded as relatively benign. \mcc{Elaborate on potential counterarguments or caveats?}

            Even if the strictly physiological effects of the disease itself are limited, additional costs may be incurred due to the costs associated with etiological uncertainty. This is true both in terms of anxiety and in terms of use of medical resources. Indeed, some authors who broadly accept the ``benign, self-limiting'' framing regard ``unnecessary hospitalization'' while etiology is determined as a primary cost of viral meningitis\cite{balada2019cost}. \mgc{Viral and bacterial meningitis are relatively easily differentiated at a clinical level based on presentation followed by evaluation of the CSF (cerebral spinal fluid – obtained with a spinal tap).  Patients with bacterial meningitis tend to appear to be much sicker, and, with meningococcal meningistis, they may have skin purpura.  If there is a suspicion of bacterial meningitis, antibiotics are started immediately, as time till initiation of antibiotics can have a profound impact on long-term outcome.  Viral meningitis generally don’t look that sick – and it is a clinical call as to whether to start antibiotics while waiting for lab tests; generally if viral meningitis is suspected, clinicians will hold on antibiotics pending results of the CSF evaluation.  CSF in bacterial meningitis is characaterized by a high WBC count, primarily polymorphonuclear (PMN) cells, with a low glucose.  These numbers can be obtained rapidly, and, if findings are consistent with bacterial meningitis, antibiotics are clearly indicated, pending culture results.  CSF from patients with viral meningitis has few white cells, almost all lymphocytes, and glucose levels are normal.  These findings, together with clinical presentation, are sufficient to rule out bacterial meningitis – although cultures of CSF will be sent, just in case.: This cost will generally be accompanied by the costs associated with unnecessary antibiotic use---in this case, unlike that of fever, arguably unavoidable, due to the rapid progression and high mortality characteristic of bacterial meningitis.} One paper\cite{hasbun2019epidemiology} specifically notes the median duration of hospital stay for viral meningitis is 2 days, ``consistent with the practice of awaiting CSF bacterial cultures.''

            The most common methods of etiological diagnosis require the use of lumbar puncture (commonly known as a ``spinal tap''), and this incurs costs of its own, not only in the performance of the procedure (and the time of qualified medical personnel required to perform it) and subsequent tests, but also in subsequent complications---both rare, but serious ones, and common ones such as post-dural puncture headache, which has been described as ``searing, and spreading like hot metal.''\cite{weir2000sharp}

        \subsection{Encephalitis (and meningoencephalitis)}
            \label{cost-encephalitis}
            \mct{I am aware that this entire section needs citations; they are on their way, as is filling in the blanks in it.}

            The cost of encephalitis naturally includes many of the costs described above, but often at a higher level. For example, both meningitis and encephalitis cases often receive empirical treatment with antibiotics, but encephalitis cases---including those that have been established to be viral, but without a specific pathogen identified---often receive empirical treatment with antivirals targeted at herpes virus, such as acyclovir. This is rarer in meningitis cases, due to the far better prognosis of untreated herpetic meningitis, compared to untreated herpetic encephalitis. In the case of encephalitis caused by La Crosse Virus, this is exacerbated by the fact that La Crosse Virus encephalitis frequently exhibits a number of neurological symptoms that have traditionally been regarded as high specific for herpetic encephalitis.\mcc{Give a few details.} As La Crosse Virus is a member of the same serogroup as KEYV, it is entirely possible that this is true of KEYV as well. This is a possible further source of underdiagnosis, for these and other California encephalitis serogroup viruses.

            %Enceph Death rate: (.138*.089+.057*.038+.004*.046+.005*.078+.004*.048+.015*.077+.032*.06+.121*.038)/(.138+.057+.004+.005+.004+.015+.032+.121) = 0.06086968085106384
            %Enceph death rate w/o herpes: (.057*.038+.004*.048+.015*.077+.032*.06+.121*.038)/(.057+.004+.015+.032+.121) =  0.043803493449781655
            %Mening death rate: 

            In addition, death, other serious complications during the acute phase, and long-lasting sequelae are much more common for encephalitis than for meningitis, not only for herpesviruses, but in general. Most prominently, the overall mortality rate for hospitalized viral encephalitis cases is just over 6\%\cite{george2014encephalitis}, compared to an overall mortality rate for hospitalized viral meningitis cases of 0.6\%\cite{holmquist2008meningitis}. Partly, this is due to the high mortality rate of herpetic encephalitis, but even if all (diagnosed) herpesviral encephalitis cases are excluded from consideration (including varicella zoster virus, cytomegalovirus, etc., and not just herpes simplex viruses 1 and 2), the mortality rate is still approximately 4.4\%, or over 7 times the mortality rate of \textit{all} viral meningitis cases, including those due to herpesviruses.

            \mcc{Seizures, etc.---fill in.}
            
            \mcc{Long-lasting complications, for La Crosse Virus in particular---fill in.}
            
            \mcc{Perhaps talk about uncertainty, swinging both ways---severe LCVE can mimic HSVE, far more serious; mild can mimic EVE, far less. Perhaps also a discussion of resolution or lack thereof of neurological deficits, from the more recent paper?}

    \section{Conclusion}
        \label{conclusion}
        Meningitis and encephalitis are serious public health burdens in the contemporary United States. In Florida, as in the US as a whole, many cases of meningitis and encephalitis never have a specific etiology diagnosed, and there is reason to believe that a substantial fraction of these may be due to mosquito-transmitted viruses. KEYV is a mosquito-transmitted virus that is closely related to several other viruses that are known to produce infections in humans that are usually asymptomatic or characterized only by non-specific symptoms such as fever, but that can also cause meningitis or encephalitis with low but non-trivial frequency. There is circumstantial evidence implicating KEYV itself in two encephalitis cases from the 1960s, and serological evidence indicating that 20\% or more of Tampa Bay area residents may have been infected with KEYV on one or more occasions. Together, these facts suggest that KEYV may be an underappreciated cause of meningitis and encephalitis cases, and consequent costs to individuals and society, in the state of Florida.

        %mening: undiag + ``viral'' (only) = + .172 + .459 = .631
        %encep: undiag + ``viral'' (only) = .349 + .121 = .47
        %(.47 * 7.3 + .631 * 24.1) / 1e5  * 327.2e6 = 60983.8632 ~= 61,000

        There are many questions that remain to be answered, in order to properly assess the costs imposed by KEYV. Which of these is most urgent to address depends on the perspective from which one examines the question. From a public health perspective, the most important question is likely to be how many of the roughly $61,000$ hospitalizations that occur each year due to meningitis and encephalitis with an undiagnosed etiology\cite{holmquist2008meningitis,george2014encephalitis}, or with an etiology that is only diagnosed as ``viral,'' are in fact due to KEYV. \mgc{The methods used by Lednicky et al are really only available at a research level.  What is becoming increasingly available commercially are multiplex PCR systems that allow identification of 20 or 30 different pathogens from a clinical sample – may want to look futher into this, and see what panels are available for meningitis or encephalitis.  So what would be needed would be for companies making these panels to include KEYV in the panels.  Which is a strong motivation for getting this review out – to push the folks making diagnostics to consider KEYV as a possible etiology in cases for which no other diagnosis has been made.} If the techniques used by Lednicky \textit{et al.}\cite{lednicky2018keystone} in the case discussed in Section \ref{recent-case} can be sufficiently refined and automated to be used routinely in cases of meningitis or encephalitis in which other approaches fail to provide a diagnosis---or to provide a diagnosis more specific than ``probably viral''---this would provide a systematic way to address not only this question, but also the analogous questions for a whole host of other viruses. More immediately, routine testing of ``viral NOS'' meningitis and encephalitis cases for (additional) clades of viruses that are already known to contain multiple species that can produce meningitis and encephalitis, such as the California encephalitis serogroup, could enhance our understanding of the relative importance of multiple viruses at once, aid in surveillance, and provide more accurate prognoses. This could be done, for example, by performing (RT-)PCR with primers for sequences that are conserved within the clade of interest, with further refinement upon obtaining a positive result.
%        Due to the progressive refinement approach involved in those techniques, beginning with unbiased PCR and RT-PCR, followed by focused confirmation of any viruses detected, this could easily be used simultaneously to also address how many of those cases are actually caused by La Crosse Virus infection---which is recognized to be a significant cause of pediatric encephalitis in the United States, but is still believed to be substantially underdiagnosed---as well as by other, unrelated viruses.

        From the perspective of a disease ecologist or transmission modeler, the most pressing question is how much the different routes of transmission contribute to maintaining endemic transmission of KEYV in Florida. The first step to addressing this question is to determine the probability of both vertical and horizontal transmission by female \atl\ that were themselves either vertically or horizontally infected. If results are informative, this might also be extended to similar studies on transmission probabilities for \alb, and potentially for other mosquito species as well.

        Given the uncertainties that exist about the public health importance of KEYV, it is understandable that funders may be reluctant to expend limited resources investigating it further. However, this must be balanced against the substantial human health impact that circumstantial evidence suggests that KEYV may have, taking into account the natural human tendency to see what one expects to see. Moreover, as noted in the paragraphs above, there is much research that could be done that would greatly increase our understanding both of the transmission dynamics and of the public health impact of KEYV, that would also have widespread application to other pathogens of public health significance.
\section*{Acknowledgements}
This project was funded in part a by grant from the National Institutes of Health/National Institute of General Medical Sciences (U54 GM111274).
    %\bibliography{converted-slides}
    %\addbibresource{converted-slides.bib}
    \printbibliography{}

\end{document}
%irrelevant test edit
