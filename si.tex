\documentclass[12pt]{article}
\usepackage[hidelinks]{hyperref}
\usepackage[backend=bibtex,hyperref=true,backref=true]{biblatex}
\usepackage{mathtools}
\usepackage[margin=1in]{geometry}
\DeclarePairedDelimiter{\parens}{\lparen}{\rparen}

\usepackage[english]{babel}
\usepackage[dvipsnames]{xcolor}
%\usepackage{hyperref}
%\newcommand{\cjh}[1]{\textcolor{blue}{(cjh: #1)}}
%\newcommand{\stupid}[1]{#1}
%\newcommand{\Ae}{\stupid{\textit{Ae.\ aegypti}}}

\newcommand{\alb}{\textit{Ae.\ albopictus}}
\newcommand{\atl}{\textit{Ae.\ atlanticus}}
\newcommand{\eg}{\textit{e.g.}}
\newcommand{\ie}{\textit{i.e.}}

\newcommand{\cjh}{\textcolor{blue}{cjh}}
\newcommand{\tjh}{\textcolor{red}{tjh}}
\newcommand{\jal}{\textcolor{green}{jal}}
\newcommand{\jgm}{\textcolor{purple}{jgm}}

\newcommand{\msg}[3]{(#1 $\rightarrow$ #2: #3)}

\newcommand{\mcc}[1]{\msg\cjh\cjh{#1}}
\newcommand{\mct}[1]{\msg\cjh\tjh{#1}}
\newcommand{\mtc}[1]{\msg\tjh\cjh{#1}}
\newcommand{\mtt}[1]{\msg\tjh\tjh{#1}}
\newcommand{\mjc}[1]{\msg\jal\cjh{#1}}
\newcommand{\mgc}[1]{\msg\jgm\cjh{#1}}

\title%[Keystone Virus in Florida]
{Ecology and Public Health Burden of Keystone Virus in Florida \\ Supplementary Information}

\author%[chrishen@umich.edu] % (optional, use only with lots of authors)
{Christopher J. Henry, John A. Lednicky, J. Glenn Morris, Thomas J. Hladish}

\date{\today}

%\bibliographystyle{plain}
\addbibresource{keystone.bib}

\begin{document}
    \maketitle
    
    \section{Pokomoke Cypress Swamp horizontal transmission}
        Orally-infected \atl\ were minimally effective at transmitting Keystone to suckling mice they fed upon: Out of 82 bites, only 1 produced a lethal infection (the primary endpoint considered; Keystone infection has a high lethality in suckling mice); 50 of the surviving mice were tested for antibodies, and only 7 were positive.\cite{watts1988maintenance} \mjc{Be careful. Suckling mice, which do not have a fully formed immune system, may not respond well.   So be sure to add more info to explain this experiment...or people will challenge the validity of your statement.} Evidence from previous dose-response experiments suggests that these 7 seropositive mice likely received very low doses of KEYV, and would have been minimally viremic at most. In contrast, 33 out of 46 mosquitoes \mgc{In a modern context, this type of work is likely to be considered questionable or wrong: infected by intrathoracic innoculation} produced lethal infections in mice they were fed on\cite{watts1988maintenance}, with 6 of the remaining 13 producing seroconversion in the fed-upon mice. Given this discrepancy, and evidence of rapid infection of sentinel rabbits exposed during seasons of peak \atl\ activity\cite{jennings1968california,leduc1978natural}, this raises the possibility that transovarially-infected (vertically-infected) \atl\ may be substantially more efficient transmitters than horizontally-infected \atl. \mjc{This explanation may not be viewed favorably in a modern context. What matters is whether the virus is in the salivary glands, and also, the amount of virus therein.} \mgc{I agree – be careful with this statement – concepts are somewhat out of date.}
            
            %\mcc{Possibly stick a mention of the TOT results in watts1988maintenance here}

    \section{Albopictus horizonal transmission}

        Grimstad \textit{et al.} fed 37 uninfected, mature female \alb\ on a mixture of KEYV and rabbit blood, allowed 14+ days for an incubation period, re-fed each on a suckling mouse, and ground up their midguts and heads (separately) and tested for the presence of KEYV. Transmission to mice was assessed by observing them for signs of disease, euthanizing any who showed such signs, and then testing their brains for signs of KEYV. The researchers found that 31 of mosquitoes developed a disseminated infection, and another 3 were infected at the midgut level, but that none of them transmitted virus to the mice they fed on. Thus, the result of this study was that none of the 34 horizontally infected \alb\ produced an infection in suckling mice that was lethal or symptomatic enough to be observed by the researchers.

        As noted in Section \ref{pokomoke-results}, Watts \textit{et al.} horizontally infected 82 \atl, re-fed each on a suckling mouse, and observed only one lethal infection. When they tested 50 of the 81 remaining mice for antibodies to KEYV, 7 were positive, suggesting a possible mild or asymptomatic infection; if those 50 mice were a representative sample of the 81, we can estimate that approximately 11 of the 81 mice that did not experience a lethal infection may have experienced a non-lethal infection. We cannot know what fraction, if any, of these mice displayed sufficient signs of morbidity that they would have been euthanized had the protocol of Grimstad \textit{et al.} been used instead. If there would have been some, we also cannot know what fraction of those, if any, would have have detectable virus in their brains, given that there is evidence that not all antibody-producing infections necessarily result even in detectable virus in the blood\cite{watts1988maintenance}. So this result may reasonably be considered equivalent to a result of somewhere from 1 to 12 detected transmissions out of 82 opportunities using the methods of Grimstad \textit{et al.}, without certainty as to where in that range it lies. Comparing this range of possible results to Grimstad \textit{et al.}'s result of 0 detected transmissions out of 34 opportunities using a Fisher's exact test, we find that the p-value ranges anywhere from 1 (i.e., no evidence at all that the probability of transmission is different) at an assumed 1 detected transmission for \atl, down to 0.0174 (statistically significant, but not highly significant) for an assumed 12 detected transmissions; the conventional cutoff of $p < 0.05$ is achieved for an assumed 10 detected transmissions (i.e., 9 out of a probable 11 non-lethal infections resulting in both sufficient morbidity to be observed by the researchers, and sufficient virus in the brain to be detected), but not at an assumed 9. In any event, this is a p-value for whether the transmission rates differ \textit{at all}, not whether \textit{Ae.~albopictus}'s transmission rate is so much lower as to make it effectively irrelevant as a vector.



    \printbibliography{}

\end{document}
