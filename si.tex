\documentclass[12pt]{article}
\usepackage[hidelinks]{hyperref}
\usepackage[backend=bibtex,hyperref=true,backref=true]{biblatex}
\usepackage{mathtools}
\usepackage[margin=1in]{geometry}
\DeclarePairedDelimiter{\parens}{\lparen}{\rparen}

\usepackage[english]{babel}
\usepackage[dvipsnames]{xcolor}
%\usepackage{hyperref}
%\newcommand{\cjh}[1]{\textcolor{blue}{(cjh: #1)}}
%\newcommand{\stupid}[1]{#1}
%\newcommand{\Ae}{\stupid{\textit{Ae.\ aegypti}}}

\newcommand{\alb}{\textit{Ae.\ albopictus}}
\newcommand{\atl}{\textit{Ae.\ atlanticus}}
\newcommand{\eg}{\textit{e.g.}}
\newcommand{\ie}{\textit{i.e.}}

\newcommand{\cjh}{\textcolor{blue}{cjh}}
\newcommand{\tjh}{\textcolor{red}{tjh}}
\newcommand{\jal}{\textcolor{green}{jal}}
\newcommand{\jgm}{\textcolor{purple}{jgm}}

\newcommand{\msg}[3]{(#1 $\rightarrow$ #2: #3)}

\newcommand{\mcc}[1]{\msg\cjh\cjh{#1}}
\newcommand{\mct}[1]{\msg\cjh\tjh{#1}}
\newcommand{\mtc}[1]{\msg\tjh\cjh{#1}}
\newcommand{\mtt}[1]{\msg\tjh\tjh{#1}}
\newcommand{\mjc}[1]{\msg\jal\cjh{#1}}
\newcommand{\mgc}[1]{\msg\jgm\cjh{#1}}

\title%[Keystone Virus in Florida]
{Ecology and Public Health Burden of Keystone Virus in Florida \\ Supplementary Information}

\author%[chrishen@umich.edu] % (optional, use only with lots of authors)
{Christopher J. Henry, John A. Lednicky, J. Glenn Morris, Thomas J. Hladish}

\date{\today}

%\bibliographystyle{plain}
\addbibresource{keystone.bib}

\begin{document}
    \maketitle
    
    \section{Pokomoke Cypress Swamp horizontal transmission}
        Orally-infected \atl\ were minimally effective at transmitting Keystone to suckling mice they fed upon: Out of 82 bites, only 1 produced a lethal infection (the primary endpoint considered; Keystone infection has a high lethality in suckling mice); 50 of the surviving mice were tested for antibodies, and only 7 were positive.\cite{watts1988maintenance} \mjc{Be careful. Suckling mice, which do not have a fully formed immune system, may not respond well.   So be sure to add more info to explain this experiment...or people will challenge the validity of your statement.} Evidence from previous dose-response experiments suggests that these 7 seropositive mice likely received very low doses of KEYV, and would have been minimally viremic at most. In contrast, 33 out of 46 mosquitoes \mgc{In a modern context, this type of work is likely to be considered questionable or wrong: infected by intrathoracic innoculation} produced lethal infections in mice they were fed on\cite{watts1988maintenance}, with 6 of the remaining 13 producing seroconversion in the fed-upon mice. Given this discrepancy, and evidence of rapid infection of sentinel rabbits exposed during seasons of peak \atl\ activity\cite{jennings1968california,leduc1978natural}, this raises the possibility that transovarially-infected (vertically-infected) \atl\ may be substantially more efficient transmitters than horizontally-infected \atl. \mjc{This explanation may not be viewed favorably in a modern context. What matters is whether the virus is in the salivary glands, and also, the amount of virus therein.} \mgc{I agree – be careful with this statement – concepts are somewhat out of date.}
            
            %\mcc{Possibly stick a mention of the TOT results in watts1988maintenance here}

    \printbibliography{}

\end{document}
